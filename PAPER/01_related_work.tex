\section{Previous Work}

\textbf{Graphical Perception.} Cleveland and McGill~\cite{cleveland_mcgill} introduced the concept of \emph{graphical perception} and investigated how different visual attributes and encodings are perceivable by humans. They define \emph{elementary perceptual tasks} as mental-visual stimuli to understand encodings in visualizations. 



\cite{HeerBostock2010} Heer and Bostock, CHI 2010 --- Crowdsourcing Graphical Perception: Using Mechanical Turk to Assess Visualization Design


\cite{Wang_linegraph_vs_scatterplot} Comparing linegraph vs. scatterplot


\cite{mckenzie_2d_3d} \cite{forsberg2009comparing_3d_vector} \cite{laidlaw_2d_vector} 2D vs. 3D Evaluation (vector fields)


\cite{kindlmann2002color} \cite{rheingans1992color} \cite{ware1988color} \cite{Rogowitz2001_colormaps} color maps

2d vs. 3d + color \cite{borkin2011arteries}

black hat vis \cite{heer2017blackhat}

timeseries \cite{herr2009timeseries}

munzner \cite{munzner2015visualization}

open vs. closed shaped \cite{open_vs_closed_shapes}

visualization ranking \cite{harrison2014_webers_law_rank}

\textbf{Visual Cortex Inspired Machine Learning.} LeNet etc and work from THomas Serre



\textbf{Computing Perception.}

Pineo et al.~\cite{Pineo2012_computational_perception} present a method to automatically evaluate and optimize visualizations using a computational model of human vision, based on a neural network simulation of the early perceptual processing in the retina and primary visual cortex. [JT] Copied from their abstract.

