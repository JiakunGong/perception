\section{Previous Work}

\paragraph{Graphical Perception} Cleveland and McGill~\cite{cleveland_mcgill,cleveland1985graphical} coin the phrase \emph{graphical perception} to describe how different visual attributes and encodings are perceived by humans. They define \emph{elementary perceptual tasks} as mental-visual stimuli to understand encodings in visualizations, and declare a ranking based on their perceptual difficulty. From these definitions, the authors propose and perform different experiments such as the \emph{position-angle} experiment which compares bar charts and pie charts, the \emph{position-length} experiment where users judge relations between encoded values in grouped and divided bar charts, and the \emph{bars-and-framed-rectangles} experiment to evaluate Weber's law \cite{harrison2014_webers_law_rank} using the proportional relation between an initial distribution density and perceivable change.

Heer and Bostock later reproduced the Cleveland-McGill experiments via crowd-sourcing on Mechanical Turk~\cite{HeerBostock2010}, with similar results. Harrison \textit{et al.}~\cite{harrison2013influencing} repeated the Cleveland-McGill experiments while observing viewer emotional states, again with similar results. Our experimental setup again reproduces the Cleveland-McGill experiments, but instead of judging human perception, we judge machine perception using convolutional neural networks. While we focus on Cleveland and McGill's work from the mid 1980s due to their repeated reproduction, many other works also investigate human perception to visual encoding~\cite{bertin1967semiologie,treisman1988feature,carpendale2003considering,wilkinson2006grammar,widgor_perception2007,munzner2015visualization}.

%Interesting are also the rankings of correlation visualization using Weber's law~\cite{harrison2014_webers_law_rank}. This law defines the proportional relation between the initial distribution density and perceivable change. In this paper, we investigate with a simple experiment whether this holds for convolutional neural networks

%\textbf{Comparing Visual Encodings.} Different visual encodings have advantages or disadvantages and the community does a great job comparing them. Higher-level comparisons include 2D versus 3D vector field and rendering studies~\cite{mckenzie_2d_3d,forsberg2009comparing_3d_vector,laidlaw_2d_vector,borkin2011arteries}, timeseries~\cite{herr2009timeseries} and scatterplots~\cite{tremmel1995visual,Wang_linegraph_vs_scatterplot}. Lower-level experiments target - besides others - open versus closed encodings~\cite{open_vs_closed_shapes}, and several evaluations of color space~\cite{ware1988color,rheingans1992color,Rogowitz2001_colormaps,kindlmann2002color}. While we investigate lower-level visual encodings in this work, we delay colorspace experiments for future work.

\paragraph{Computational Visualization Understanding} Pineo \textit{et al.}~\cite{Pineo2012_computational_perception} create a computational model of human vision based on neural networks. Their simulations show that understanding visualization triggers neural activity in high-level areas of cognition, with the authors suspecting that this activity is supported by low-level neurons performing elementary perceptional tasks. Other work tries to parse infographics by finding higher-level saliency models~\cite{bylinskii2016should}, or by extracting text or key visual elements from visualizations~\cite{diagram_understanding,kembhavi2016diagram,zoya_text_visual_tags}. However, these works do not investigate computational understanding of elementary perceptual tasks such as curvature, lengths, or position, which are the building blocks of visualization.

%CNNs trained on natural images have low-level `neurons' which activate on similar stimuli to early neurons in the human visual cortex, e.g., Gabor filter-like stimuli \misscite.

\paragraph{Visual-cortex-inspired Machine Learning} The human visual cortex allows us to recognize objects in the world seemingly without effort (though few remember their infancy). This visual system is organized in layers, which inspired computational classifiers based on multilayer neural networks. Fukushima and Miyake developed the early Neocognitron quantitative model~\cite{fukushima1982neocognitron}, which ultimately led to the work of Hinton, Bengio, and LeCun~\cite{lecun2015deep} and today's GPU-powered \emph{deep} neural networks. Such networks exist with many architectures. For this paper, we select the traditional LeNet-5 which was designed to recognize hand-written digits with 5 layers~\cite{lenet}, the VGG19 network with 16 convolutional layers~\cite{simonyan_very_deep2014}, and the Xception network with 36 convolutional layers ~\cite{xception}. This allows us to compare archtitectures with different depths.
%Object detection is extremely difficult and therefore is especially impressive as light intensities can change by levels of magnitude and contrast between foreground and background is so often low. 
%In addition, the visual scene changes every time the human body or human eyes move. 
%This visual system exhibits a very noisy structure but because it is organized by layers it has inspired the mathematical theory of multilayer neural networks. 

%In 1962 Hubel and Wiesel were the first to begin studying the visual cortex from the standpoint of a neuroscientist. Their experimental findings on cats and macaque monkeys suggested a hierarchy of cells with increasing complexity which was then later transferred to the hierarchical model of different layers. Twenty years later, this insight was translated to the Neocognitron quantitative model, by Fukushima and Miyake, which ultimately led to the important work of Hinton, Bengio, and LeCun in the 1980s. Their work on stochastic gradient descent approximation, and the availability of faster computer hardware then led to today’s deep learning networks. In the last decade, this field has exhibited rapid growth, constant evolution, and new applications in various domains.


%The reported classification performance is superior to that of humans and the question in regards to their functionality opens an interesting research topic.
%Visual cortex inspired machine learning classifiers exist with many different architectures. We select the traditional \emph{LeNet-5}~\cite{lenet} which was designed to recognize hand-written digits, the VGG19~\cite{simonyan_very_deep2014} classifier with 16 convolutional layers, and the Xception~\cite{xception} classifier with 36 convolutional layers. 
%
%What is remarkable is that even though current machine learning models do not resemble the complexity of its biological pendant, they inherently are able to generalize but only after learning thousands of examples. 
%Neural networks trained on one specific task can be used to perform detection or segmentation of, seemingly, unrelated objects with relatively minor retraining~\cite{transfer_learning}. 
%
%
%We think that the biological inspiration of modern convolutional neural networks yields the evaluation of principles of human perception with computers.
%
%
%MLP, LeNet, VGG, ImageNet, XCeption, ResNets etc and work from THomas Serre
%
%
%- https://link.springer.com/chapter/10.1007/978-3-642-14600-8_46
%
%- https://github.com/tidyverse/ggplot2/wiki/Recommended-Reading
%
%- last year's VADL workshop: https://vadl2017.github.io/


